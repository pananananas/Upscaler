\chapter{Wstęp}

W dobie cyfryzacji i rosnącej roli mediów wizualnych, jakość obrazu staje się kluczowym aspektem w wielu dziedzinach, zastosowania sięgają od rozrywki po medycynę. W odpowiedzi na te potrzeby, niniejsza praca inżynierska koncentruje się na opracowaniu aplikacji webowej, która wykorzystuje zaawansowane techniki przetwarzania obrazu w celu zwiększenia ich rozdzielczości.

Celem tej pracy jest zaprojektowanie i implementacja aplikacji, która z wykorzystaniem algorytmów DWSR \cite{guo2017deep} oraz ESRGAN \cite{wang2018esrgan} umożliwia znaczące polepszenie jakości obrazów. Wybór tych technologii podyktowany jest ich nowoczesnością i skutecznością.

% Motywacją do podjęcia tego tematu są rosnące potrzeby szerokiej dostępności narzędzi umożliwiających powiększanie rozdzielczości obrazów.

Celem pracy jest stworzenie nie tylko narzędzia do zwiększania rozdzielczości obrazów, ale przede wszystkim wdrożenie intuicyjnego interfejsu użytkownika umożliwiającego łatwe wykorzystanie algorytmów rozwiązujących problem super-rozdzielczości. W tym celu wykorzystane zostaną nowoczesne technologie webowe, które umożliwią dostęp do aplikacji z poziomu przeglądarki internetowej.

Układ pracy jest następujący. W rozdziale \ref{chap:podstawy_teoretyczne}. przedstawione zostały zagadnienia teoretyczne związane z tematem pracy. Kolejne rozdziały szczegółowo opisują zaimplementowane algorytmy: rozdział \ref{chap:DWSR}. opisuje algorytm DWSR, a rozdział \ref{chap:ESRGAN}. algorytm ESRGAN. Rozdział \ref{chap:app}. opisuje aplikację webową - narzędzie do porównywania wyników działania algorytmów. Rozdział \ref{chap:porownanie_algorytmow}. zawiera zestawienie zaimplementowanych metod. Rozdział \ref{chap:podsumowanie}. podsumowuje całość.