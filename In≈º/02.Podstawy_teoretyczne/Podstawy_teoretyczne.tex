\chapter{Podstawy teoretyczne}
\section{Definicja super-rozdzielczości}
% \subsection{Wyjaśnienie pojęcia super-rozdzielczości w kontekście przetwarzania obrazów i jej znaczenia dla tej dziedziny.}
\section{Przegląd metod powiększania obrazów}
% \subsection{Omówienie różnych technik używanych do zwiększania rozdzielczości obrazów, ich zalet i ograniczeń. Od technik takich jak interpolacja liniowa do uczenia maszynowego.}
\section{Wprowadzenie do głębokiego uczenia się w przetwarzaniu obrazów}
\subsection{Wprowadzenie do roli i zastosowań głębokich sieci neuronowych w przetwarzaniu i analizie obrazów.}
\section{Wstęp do funkcji falkowych}
\subsection{Omówienie funkcji falkowych, opis do czego to narzędzie służy począwszy od transformaty Fouriera i jej ograniczeń, w jaki sposób funkcje falkowe rozwijają FFT, przedstawienie działania.}