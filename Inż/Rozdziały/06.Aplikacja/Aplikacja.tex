\chapter{Aplikacja webowa do powiększania rozdzielczości obrazów}

Podczas korzystania z Internetu miałem kilka sytuacji w których potrzebowałem narzędzia, które pozwoli mi na powiększenie rozdzielczości obrazów. Stron internetowych tego typu jest wiele, sporo aplikacji do edycji zdjęć umożliwia powiększenie rozdzielczości obrazów, przykładem może być \textbf{Photoshop} i inne.

Korzystając z rozwiązań ogólnodostępnych zauważyłem, że darmowe aplikacje nie gwarantują wysokiej jakości obrazów wyjściowych, zaś bardziej rozbudowane rozwiązania są płatne lub mają dostępne tylko jedno powiększenie obrazu na dobę. 

\section{Założenia projektu}

Chciałbym, żeby aplikacja była prosta w obsłudze, maksymalnie 


\begin{figure}[ht]
    \centering
    \begin{minipage}[t]{0.99\linewidth}
        \includegraphics[width=\linewidth]{Rozdziały/06.Aplikacja/Obrazy/kursor-default.png}  
        \caption{Widok strony głównej aplikacji}
        \label{fig:image80}
    \end{minipage}
\end{figure}


\begin{figure}[ht]
    \centering
    \begin{minipage}[t]{0.99\linewidth}
        \includegraphics[width=\linewidth]{Rozdziały/06.Aplikacja/Obrazy/result-siano.png}  
        \caption{Widok prezentacji wyników}
        \label{fig:image81}
    \end{minipage}
\end{figure}




\section{Projektowanie aplikacji}


Wytłumaczenie wyboru określonych technologii i narzędzi użytych do stworzenia aplikacji webowej.




\subsection*{Projekt interfejsu użytkownika}


Omówienie procesu projektowania interfejsu użytkownika, w tym wytycznych ergonomii i użyteczności.



\section{Wybór narzędzi i technologii}

K-means algorytm do detekcji kolorów wystepujacych na obrazie.


\section{Implementacja aplikacji}


Opis technicznego procesu integracji wybranych algorytmów z aplikacją, wraz z napotkanymi wyzwaniami.

\section{Integracja algorytmów DWSR i ESRGAN}


\section{Wdrożenie i utrzymanie aplikacji}

Omówienie procesu wdrożenia gotowej aplikacji oraz planów dotyczących jej przyszłego utrzymania i aktualizacji.
