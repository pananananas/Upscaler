\chapter{Porównanie algorytmów ESRGAN i DWSR} \label{chap:porownanie_algorytmow}

Skoro mamy narzędzie pozwalające korzystać algorytmów \textbf{ESRGAN} i \textbf{DWSR} do zadania super-rozdzielczości, to warto porównać te metody. W tym rozdziale zostaną przedstawione kryteria porównawcze, na podstawie których zostaną ocenione oba algorytmy. Następnie zostanie przeprowadzona analiza wydajności obu metod w różnych warunkach. Na koniec zostanie oceniona jakość obrazów generowanych przez oba algorytmy, uwzględniając różne aspekty jakości wizualnej.


\section{Kryteria porównawcze}


Ustalenie kryteriów, które będą stosowane do oceny i porównania skuteczności i efektywności algorytmów super rozdzielczości.


\cite{wang2018esrgan}


\section{Analiza wydajności}


Bezpośrednie porównanie wydajności obu metod w różnych warunkach, bazujące na ustalonych kryteriach.



\section{Jakość odtwarzania obrazów}


Ocena jakości obrazów generowanych przez oba algorytmy, uwzględniając różne aspekty jakości wizualnej.



\section{Ograniczenia i wyzwania}


Dyskusja na temat ograniczeń obu metod i potencjalnych wyzwań w ich stosowaniu.