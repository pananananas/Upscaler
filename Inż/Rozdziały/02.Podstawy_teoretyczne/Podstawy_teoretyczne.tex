\chapter{Podstawy teoretyczne}

Celem rozdziału jest przedstawienie podstawowych definicji, wytłumaczenie aparatu matematycznego oraz metod wykorzystywanych w algorytmach na których skupia się praca. Dodatkowo ma on na celu ułatwienie dalszego czytania poprzez zapoznanie czytelnika z przyjętymi konwencjami, oznaczeniami oraz symbolami, które mogą pojawić się w kolejnych rozdziałach. 


\section{Definicja super-rozdzielczości}




\section{Przegląd metod powiększania obrazów}


Omówienie technik zwiększania rozdzielczości obrazów od interpolacji liniowej do uczenia maszynowego


\section{Wprowadzenie do głębokiego uczenia się w przetwarzaniu obrazów}


Wprowadzenie do roli i zastosowań głębokich sieci neuronowych w przetwarzaniu i analizie obrazów


\section{Wstęp do funkcji falkowych}


Omówienie funkcji falkowych, opis do czego to narzędzie służy począwszy od transformaty Fouriera i jej ograniczeń, 
w jaki sposób funkcje falkowe rozwijają FFT, przedstawienie działania.

