
%opcje klasy dokumentu mgr.cls zostały opisane w dołączonej instrukcji, patrz plik <manual.pdf>

%poniżej deklaracje użycia pakietów, usunąć to co jest niepotrzebne
\usepackage{polski}       %przydatne podczas składania dokumentów w
%j. polskim 
%\usepackage[polish]{babel} %alternatywnie do pakietu
%polski, wybrać jeden z nich
%\usepackage[latin2]{inputenc} %kodowanie znaków, zależne od systemu
%\usepackage[T1]{fontenc} %poprawne składanie polskich czcionek

%pakiety do grafiki
\usepackage{graphicx}
\usepackage{subfigure}
\usepackage{psfrag}

%pakiety dodające dużo dodatkowych poleceń matematycznych
\usepackage{amsmath}
\usepackage{amsfonts}

%pakiety wspomagające i poprawiające składanie tabel
\usepackage{supertabular}
\usepackage{array}
\usepackage{tabularx}
\usepackage{hhline}

%pakiet wypisujący na marginesie etykiety równań i rysunków
%zdefiniowanych przez \label{}, chcąc wygenerować finalną wersję

\usepackage{showlabels}


\usepackage{microtype}


%definicje własnych poleceń
\newcommand{\R}{I\!\!R} %symbol liczb rzeczywistych, działa tylko w
                        %trybie matematycznym
\newtheorem{theorem}{Twierdzenie}[section] %nowe otoczenie do
                                           %składania twierdzeń

%dane do złożenia strony tytułowej
\title{Aplikacja webowa zwiększająca \\rozdzielczość obrazów}
\engtitle{Master thesis title}
\author{Eryk Wójcik}
\supervisor{dr hab. inż. Andrzej Rusiecki Prof. PWr}


%\guardian{dr hab. inż. Imię Nazwisko Prof. PWr} %nie używać
%jeśli opiekun jest tą samą osobą co prowadzący pracę

% \date{2023} %standardowo u dołu strony tytułowej umieszczany jest


%poniżej definiuje sie kierunek i ewentualną specjalność
%w obecnej wersji specjalność nie jest wypisywana na stronie tytułowej

\field{Automatyka i Robotyka (AIR)}
\specialisation{ART (ART)}
