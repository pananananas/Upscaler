\usepackage{float}
%opcje klasy dokumentu mgr.cls zostały opisane w dołączonej instrukcji, patrz plik <manual.pdf>

%poniżej deklaracje użycia pakietów, usunąć to co jest niepotrzebne
\usepackage{polski}       %przydatne podczas składania dokumentów w
%j. polskim 
%\usepackage[polish]{babel} %alternatywnie do pakietu
%polski, wybrać jeden z nich
%\usepackage[latin2]{inputenc} %kodowanie znaków, zależne od systemu
%\usepackage[T1]{fontenc} %poprawne składanie polskich czcionek

%pakiety do grafiki
\usepackage{graphicx}
\usepackage{subfigure}
\usepackage{psfrag}

%pakiety dodające dużo dodatkowych poleceń matematycznych
\usepackage{amsmath}
\usepackage{amsfonts}

%pakiety wspomagające i poprawiające składanie tabel
\usepackage{supertabular}
\usepackage{array}
\usepackage{tabularx}
\usepackage{hhline}

%pakiet wypisujący na marginesie etykiety równań i rysunków
%zdefiniowanych przez \label{}, chcąc wygenerować finalną wersję

% \usepackage{showlabels}


\usepackage{microtype}


%definicje własnych poleceń
\newcommand{\R}{I\!\!R} %symbol liczb rzeczywistych, działa tylko w
                        %trybie matematycznym
\newtheorem{theorem}{Twierdzenie}[section] %nowe otoczenie do
                                           %składania twierdzeń

%dane do złożenia strony tytułowej
\title{Aplikacja webowa zwiększająca \\rozdzielczość obrazów}
\engtitle{Master thesis title}
\author{Eryk Wójcik}
\supervisor{dr hab. inż. Andrzej Rusiecki,\\Katedra Informatyki Technicznej}


%\guardian{dr hab. inż. Imię Nazwisko Prof. PWr} %nie używać
%jeśli opiekun jest tą samą osobą co prowadzący pracę

% \date{2023} %standardowo u dołu strony tytułowej umieszczany jest


%poniżej definiuje sie kierunek i ewentualną specjalność
%w obecnej wersji specjalność nie jest wypisywana na stronie tytułowej

\field{Automatyka i Robotyka (AIR)}
\specialisation{ART (ART)}


\usepackage{fontspec}
% \setmainfont{larken}                            % Font family 

\usepackage{hyperref}
\hypersetup{
  colorlinks=true,
  linkcolor=black,
  filecolor=blue,      
  urlcolor=cyan,
}

\usepackage{caption}

\captionsetup[figure]{name=Rys,labelsep=period}     % Zmienia "Rysunek" na "rys" i separator na kropkę
\renewcommand{\thefigure}{\arabic{figure}}          % Usuwa numer sekcji z numeracji rysunku

\setlength{\parindent}{0pt}                         % Usuwa tabulator w paragrafach


\usepackage{pgfplots}
\pgfplotsset{compat=newest}


\usepackage{listings}
\definecolor{clr-background}{RGB}{255,255,255}
\definecolor{clr-text}{RGB}{0,0,0}
\definecolor{clr-string}{RGB}{163,21,21}
\definecolor{clr-namespace}{RGB}{0,0,0}
\definecolor{clr-preprocessor}{RGB}{128,128,128}
\definecolor{clr-keyword}{RGB}{0,0,255}
\definecolor{clr-type}{RGB}{59, 112, 230}
\definecolor{clr-variable}{RGB}{0,0,0}
\definecolor{clr-constant}{RGB}{111,0,138} % macro color
\definecolor{clr-comment}{RGB}{0,128,0}
\definecolor{mycolor}{rgb}{0.8,0.8,0.8}
\lstset{
  xleftmargin        = 20pt,
  xrightmargin       = 20pt,
  framexleftmargin   = 20pt,
  framexrightmargin  = 20pt,
  framexbottommargin = 2pt,
  columns = flexible,
  keepspaces = true,
  showstringspaces = false,
  backgroundcolor  = \color{clr-background},
  basicstyle       = \color{clr-text}, % any text
  stringstyle      = \color{clr-string},
  identifierstyle  = \color{clr-variable}, % just about anything that isn't a directive, comment, string or known type
  commentstyle     = \color{clr-comment},
  keywordstyle     = \color{clr-type},
  tabsize   = 4,
  aboveskip = 1em,
  belowskip = 0em,
  frame     = b,
  rulecolor = \color{mycolor},
  numbers   = left,
  numbersep = 10pt,
  numberstyle = {\fontsize{9pt}{11pt}\selectfont\color{gray}},
}


\usepackage{xcolor}

%New colors defined below
\definecolor{codegreen}{rgb}{0,0.6,0}
\definecolor{codegray}{rgb}{0.5,0.5,0.5}
\definecolor{codepurple}{rgb}{0.58,0,0.82}
\definecolor{backcolour}{rgb}{1,0.99,0.95}

%Code listing style named "mystyle"
\lstdefinestyle{mystylecode}{
  backgroundcolor=\color{backcolour}, commentstyle=\color{codegreen},
  keywordstyle=\color{magenta},
  numberstyle=\tiny\color{codegray},
  stringstyle=\color{codepurple},
  basicstyle=\ttfamily\footnotesize,
  breakatwhitespace=false,         
  breaklines=true,                 
  captionpos=b,                    
  keepspaces=true,                 
  numbers=left,                    
  numbersep=5pt,                  
  showspaces=false,                
  showstringspaces=false,
  showtabs=false,                  
  tabsize=2
}

%"mystyle" code listing set
\lstset{style=mystylecode}